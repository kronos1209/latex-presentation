% beamer クラスを指定 + アスペクト比を16:9に設定(4:3 の場合は aspectration=43 を指定する)
\documentclass[aspectratio=169]{beamer}

\usepackage{luatexja}           % 日本語処理
\usepackage{luatexja-fontspec}  % フォント指定
% \setmainjfont{IPAexGothic}      % 任意の日本語フォント(WSLに入ってる前提)

\usepackage{graphicx}        % 画像挿入用
\usepackage{listings,listings-rust}        % コードブロック用
\usepackage{booktabs}      % 表の罫線用
\usepackage{caption}     % キャプションの設定

% lstings のグローバル設定
% コードブロックはすべて Rust コードとして解釈する
\lstset{language=Rust, style=mdRust}

% Beamerのテーマ設定
\usetheme{Madrid}
\setbeamertemplate{navigation symbols}{}

\title{タイトルを入力する}
\author{名前を入力する}
\institute{所属を入力する}
\date{\today}   % \today はビルドした日付として解釈される  
\begin{document}

\frame{\titlepage}

\begin{frame}{日本語テスト}
    LuaLaTeX を使った日本語対応の beamer スライドです。
\end{frame}

\begin{frame}[fragile]{コードブロック}
    おはようございます. \par
    以下は Rust のコードブロックの例です。\par
    ほげほげうげほげ\par

    % このスライド内でのみコードブロックのフォントサイズを小さくする
    \lstset{basicstyle=\tiny}

    \begin{columns}
        \begin{column}{0.5\linewidth}
            \begin{lstlisting}
#[tokio::main]
async fn main() -> std::result::Result<(), Box<dyn std::error::Error>> {
    let _guard = common::init(
        env!("CARGO_PKG_NAME").to_string(),
        env!("CARGO_PKG_VERSION").to_string(),
    );
    tonic::transport::Server::builder()
        .layer(tonic_tracing_opentelemetry::middleware::server::OtelGrpcLayer::default())
        .add_service(
            proto_definition::product::product_service_server::ProductServiceServer::new(
                product::product::Product,
            ),
        )
        .serve("0.0.0.0:8082".parse()?)
        .await?;
    Ok(())
}

\end{lstlisting}
        \end{column}
        \begin{column}{0.5\linewidth}
            \begin{lstlisting}
#[tokio::main]
async fn main() -> std::result::Result<(), Box<dyn std::error::Error>> {
    let _guard = common::init(
        env!("CARGO_PKG_NAME").to_string(),
        env!("CARGO_PKG_VERSION").to_string(),
    );
    tonic::transport::Server::builder()
        .layer(tonic_tracing_opentelemetry::middleware::server::OtelGrpcLayer::default())
        .add_service(
            proto_definition::product::product_service_server::ProductServiceServer::new(
                product::product::Product,
            ),
        )
        .serve("0.0.0.0:8082".parse()?)
        .await?;
    Ok(())
}

            \end{lstlisting}
        \end{column}
    \end{columns}

\end{frame}

\begin{frame}{図の挿入}
    \includegraphics[width=0.6\linewidth]{example-image}
\end{frame}

\begin{frame}{表の挿入}
    \begin{tabular}{lll}
        \toprule
        名前 & 職業    & 年齢 \\
        \midrule
        佐藤 & エンジニア & 28 \\
        鈴木 & デザイナー & 35 \\
        \bottomrule
    \end{tabular}
\end{frame}

\begin{frame}[fragile]{追加}
    \begin{lstlisting}
#[tokio::main]
async fn main() -> std::result::Result<(), Box<dyn std::error::Error>> {
    let _guard = common::init(
        env!("CARGO_PKG_NAME").to_string(),
        env!("CARGO_PKG_VERSION").to_string(),
    );
    tonic::transport::Server::builder()
        .layer(tonic_tracing_opentelemetry::middleware::server::OtelGrpcLayer::default())
        .add_service(
            proto_definition::product::product_service_server::ProductServiceServer::new(
                product::product::Product,
            ),
        )
        .serve("0.0.0.0:8082".parse()?)
        .await?;
    Ok(())
}
    \end{lstlisting}
\end{frame}

\end{document}
