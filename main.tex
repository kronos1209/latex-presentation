% beamer クラスを指定 + アスペクト比を16:9に設定(4:3 の場合は aspectration=43 を指定する)
\documentclass[aspectratio=169]{beamer}

\usepackage{luatexja}           % 日本語処理
\usepackage{luatexja-fontspec}  % フォント指定
% \setmainjfont{IPAexGothic}      % 任意の日本語フォント(WSLに入ってる前提)

\usepackage{graphicx}        % 画像挿入用
\usepackage{listings,listings-rust}        % コードブロック用

\usepackage{booktabs}      % 表の罫線用
\usepackage{caption}     % キャプションの設定

% Beamerのテーマ設定
\usetheme{Warsaw}
\setbeamertemplate{navigation symbols}{}
% \setbeamertemplate{footline}{}


\title{LuaLaTeX + beamer サンプル}
\author{あなたの名前}
\date{\today}

\begin{document}

\frame{\titlepage}

\begin{frame}{日本語テスト}
    LuaLaTeX を使った日本語対応の beamer スライドです。
\end{frame}

\begin{frame}[fragile]{コードブロック}
    \begin{columns}
        \begin{column}{0.5\linewidth}
            \begin{lstlisting}[language=Rust,style=mdRust]
fn main() {
    println!("こんにちは、世界!");
}
pub trait Hoge {
    fn huga(&self) -> &'static str;
}
\end{lstlisting}
        \end{column}
        \begin{column}{0.5\linewidth}
            \begin{lstlisting}[language=Rust,style=mdRust]
fn main() {
    println!("Hello, world!");
}
pub trait Hoge {
    fn huga(&self) -> &'static str;
}
\end{lstlisting}
        \end{column}
    \end{columns}

\end{frame}

\begin{frame}{図の挿入}
    \includegraphics[width=0.6\linewidth]{example-image}
\end{frame}

\begin{frame}{表の挿入}
    \begin{tabular}{lll}
        \toprule
        名前 & 職業    & 年齢 \\
        \midrule
        佐藤 & エンジニア & 28 \\
        鈴木 & デザイナー & 35 \\
        \bottomrule
    \end{tabular}
\end{frame}

\end{document}
