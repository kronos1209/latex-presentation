% サンプルを元にプロットを出力する例\usetikzlibrary {datavisualization}
\begin{frame}
    \frametitle{プロット}

    hogehoge

    ふがふが

    yahho-

    \begin{center}  % プロットをセンタリングする環境
        \begin{tikzpicture}
            \begin{axis}[
                    xlabel=Sample X,    % x ラベル
                    ylabel=Sample Y,    % y ラベル
                    % axis x line = center,  % x 軸の位置を変える
                    % axis y line = center,  % y 軸の位置を変える
                    % xmin=1,    % x 軸の描画範囲(最大)
                    % xmax=3,    % x 軸の描画範囲(最小)
                    % ymin=1,    % y 軸の描画範囲(最大)
                    % ymax=3,    % y 軸の描画範囲(最小)
                    grid=major,         % 
                    width = 10cm,       % プロットの横の長さ
                    height = 5cm        % プロットの縦の長さ
                ]
                % CSVファイルを読み込む
                \addplot table[
                        col sep=comma, % CSVファイルのカラム分割を定義
                        x=x,
                        y=y,
                    ] {../example/data/sample.csv}; % sample.csv を直接指定
            \end{axis}
        \end{tikzpicture}
    \end{center}
\end{frame}
