% beamer クラスを指定 + アスペクト比を16:9に設定(4:3 の場合は aspectration=43 を指定する)
\documentclass[aspectratio=169]{beamer}

\usepackage{luatexja}           % 日本語処理
\usepackage{luatexja-fontspec}  % フォント指定
% \setmainjfont{IPAexGothic}      % 任意の日本語フォント(WSLに入ってる前提)

% bibtex の設定(参考文献の管理する)
\usepackage[backend=biber]{biblatex}
\addbibresource{reference.bib}

\usepackage{pgfplots}
\pgfplotsset{compat=1.18}
\usepackage{minted}         %python の pygments を利用してコードハイライトを行う
% \usepackage{xcolor}
\usepackage{graphicx}        % 画像挿入用
\usepackage{listings}        % コードブロック用
\usepackage{booktabs}      % 表の罫線用
\usepackage{caption}     % キャプションの設定
\usepackage{tikz}       % 図形用
\usetikzlibrary {datavisualization}

% /style/* から
\usepackage{listings-rust}
\usepackage{tikz-uml}

\usetikzlibrary{positioning}
% lstings のグローバル設定
% コードブロックはすべて Rust コードとして解釈する
\lstset{language=Rust, style=mdRust}



% Beamerのテーマ設定
\usetheme{Madrid}
\setbeamertemplate{navigation symbols}{}   % スライドのナビゲーションバーを非表示にする 
% フッターとして[タイトル、日付、ページ番号]のみが表示されるようにする
\setbeamertemplate{footline}{
  \leavevmode%
  \hbox{%
    \begin{beamercolorbox}[wd=0.7\paperwidth,ht=2.5ex,dp=1ex,leftskip=1em]{author in head/foot}%
      \usebeamerfont{title in head/foot}\insertshorttitle
    \end{beamercolorbox}%
    \begin{beamercolorbox}[wd=0.2\paperwidth,ht=2.5ex,dp=1ex,center]{date in head/foot}%
      \usebeamerfont{date in head/foot}\insertshortdate
    \end{beamercolorbox}%
    \begin{beamercolorbox}[wd=0.1\paperwidth,ht=2.5ex,dp=1ex,right]{frame number in head/foot}%
      \usebeamerfont{page number in head/foot}\insertframenumber{} / \inserttotalframenumber
      \hspace*{1em}
    \end{beamercolorbox}%
  }%
  \vskip0pt%
}

% タイトルや著者の情報を定義する
% TODO: 自分の情報に置き換える
\title{タイトル}
\author{名前}
\institute{所属}
\date{\today}   % \today はビルドした日付として解釈される  

\begin{document}

% タイトルスライド
\frame{\titlepage}

% 目次スライド
% \section で定義したものが表示される
\begin{frame}{目次}
  \tableofcontents
\end{frame}

%% =====================================================================
%% ここからサンプルのスライド
%% 自分のスライドを作成する時は削除すればよい
\section{section を使うと目次ページに表示される}
\subsection{subsection を使うと目次ページで親セクションの下に表示される}

\section{図の挿入}
\begin{frame}{図の挿入}
    \includegraphics[width=0.6\linewidth]{example-image}
\end{frame}

\section{表の挿入}
\begin{frame}{表の挿入}
    \begin{tabular}{lll}
        \toprule
        名前 & 職業    & 年齢 \\
        \midrule
        佐藤 & エンジニア & 28 \\
        鈴木 & デザイナー & 35 \\
        \bottomrule
    \end{tabular}
\end{frame}

\section{強調ブロック}
\begin{frame}{強調ブロック}
    beamer で強調ブロックを作成することができます。
    
    色などは利用するテーマによって変わる可能性があります。

    \begin{block}{blue block}
        block env では基本的には青色のブロックを作成します.
    \end{block}

    \begin{alertblock}{reb block}
        alertblock では基本的には赤色のブロックを作成します。
    \end{alertblock}

    \begin{exampleblock}{green block}
        exampleblock では基本的は緑色のブロックを作成します。
    \end{exampleblock}

    
\end{frame}

\section{スライドのレイアウト変更}
\begin{frame}[fragile]{二分割スライド}
    おはようございます. \par
    ここで記述する文字列はスライドの横幅いっぱいまで利用して描画されます.\par

    \noindent\rule[7pt]{\linewidth}{0.4pt}
    
    % このスライド内でのみコードブロックのフォントサイズを小さくする
    \lstset{basicstyle=\tiny}

    \begin{columns}
        \begin{column}{0.5\linewidth}
            こっちは左側のコンテンツが表示されます。
        \end{column}
        \begin{column}{0.5\linewidth}
            こっちは右側のコンテンツが表示されます。
        \end{column}
    \end{columns}
\end{frame}

\section{コードブロック}
% コードブロックを入れるスライドには frame のオプションとして fragile を追加する。
% これを指定することでコードブロック内で特殊文字を利用してもそのまま出力されるようになる
\begin{frame}[fragile]{コードブロック}
    % このスライド内でのコードブロックのスタイルを変える場合は \begin{frame} 内で \lstset を呼び出す
    % 特定のコードブロックにのみ適用したい場合は下の \begin{lstlisting}[options] の options の部分で設定する
    % \lstset{basicstyle=\scriptsize}

    \begin{lstlisting}[caption=タイトル,language=Rust,style=mdRust,basicstyle=\scriptsize]
#[tokio::main]
async fn main() -> std::result::Result<(), Box<dyn std::error::Error>> {
    let _guard = common::init(
        env!("CARGO_PKG_NAME").to_string(),
        env!("CARGO_PKG_VERSION").to_string(),
    );
    tonic::transport::Server::builder()
        .layer(tonic_tracing_opentelemetry::middleware::server::OtelGrpcLayer::default())
        .add_service(
            proto_definition::product::product_service_server::ProductServiceServer::new(
                product::product::Product,
            ),
        )
        .serve("0.0.0.0:8082".parse()?)
        .await?;
    Ok(())
}
    \end{lstlisting}
\end{frame}

% xcolor のマクロでカラーを定義する
\definecolor{LightGray}{gray}{0.9}

\begin{frame}[fragile]
    \frametitle{minted パッケージを使ったコードブロック}

    \begin{minted}[
            frame = lines,  %コードブロックの上限にラインが表示される
            bgcolor=LightGray,  %コードブロックの背景を指定する
            fontsize=\scriptsize   %コードブロック内の文字サイズを指定する
        ]{Rust}
#[tokio::main]
async fn main() -> std::result::Result<(), Box<dyn std::error::Error>> {
    let _guard = common::init(
        env!("CARGO_PKG_NAME").to_string(),
        env!("CARGO_PKG_VERSION").to_string(),
    );
    tonic::transport::Server::builder()
        .layer(tonic_tracing_opentelemetry::middleware::server::OtelGrpcLayer::default())
        .add_service(
            proto_definition::product::product_service_server::ProductServiceServer::new(
                product::product::Product,
            ),
        )
        .serve("0.0.0.0:8082".parse()?)
        .await?;
    Ok(())
}
    \end{minted}

\end{frame}

\section{uml}
\begin{frame}[fragile]{uml図の挿入}
    \resizebox{\textwidth}{0.8\textheight}{
        \begin{tikzpicture}

            % インターフェース
            \umlinterface[x=0, y=3]{IPrintable}{
                + print(): void
            }

            % 親クラス
            \umlclass[x=5, y=6]{Document}{
                - content: String
                + title: String
            }{
                + getContent(): String \\   % \\ は改行を表す
                + setContent(content: String): void
            }

            % サブクラス1
            \umlclass[x=0, y=0]{PDFDocument}{
                - pdfVersion: Float
            }{
                + generate(): void
            }

            % サブクラス2
            \umlclass[x=5, y=0]{WordDocument}{
                - template: String
            }{
                + export(): void
            }

            % 依存先クラス
            \umlclass[x=10, y=0]{Printer}{
                + model: String
            }{
                + print(document: Document): void
            }

            % サブクラスへの継承
            \umlinherit{PDFDocument}{Document}
            \umlinherit{WordDocument}{Document}

            % インターフェースの実装
            \umlimpl{PDFDocument}{IPrintable}
            \umlimpl{WordDocument}{IPrintable}

            % Printer との集約
            \umluniassoc[mult=1]{Printer}{PDFDocument}
            \umluniassoc[mult=1]{Printer}{WordDocument}

            % Document → Printer への依存(点線矢印)
            \umldep{Document}{Printer}

        \end{tikzpicture}
    }
\end{frame}

\begin{frame}
    \frametitle{uml図の挿入(シーケンス図)}
    % シーケンス図はビルドが失敗する。
\end{frame}

\section{プロット}
% サンプルを元にプロットを出力する例\usetikzlibrary {datavisualization}
\begin{frame}
    \frametitle{プロット}

    hogehoge

    ふがふが

    yahho-

    \begin{center}  % プロットをセンタリングする環境
        \begin{tikzpicture}
            \begin{axis}[
                    xlabel=Sample X,    % x ラベル
                    ylabel=Sample Y,    % y ラベル
                    % axis x line = center,  % x 軸の位置を変える
                    % axis y line = center,  % y 軸の位置を変える
                    % xmin=1,    % x 軸の描画範囲(最大)
                    % xmax=3,    % x 軸の描画範囲(最小)
                    % ymin=1,    % y 軸の描画範囲(最大)
                    % ymax=3,    % y 軸の描画範囲(最小)
                    grid=major,         % 
                    width = 10cm,       % プロットの横の長さ
                    height = 5cm        % プロットの縦の長さ
                ]
                % CSVファイルを読み込む
                \addplot table[
                        col sep=comma, % CSVファイルのカラム分割を定義
                        x=x,
                        y=y,
                    ] {../example/data/sample.csv}; % sample.csv を直接指定
            \end{axis}
        \end{tikzpicture}
    \end{center}
\end{frame}


\section{参考文献の挿入}
\begin{frame}
    \frametitle{参考文献の挿入}
    \begin{itemize}
        \item 引用は \cite{einstein1905annus} のように行います。
        \item 複数引用は \cite{knuth1984texbook, turing1936computable} のように可能です。
        \item 文中に著者を出す引用は `\textcite{einstein1905annus}` のようになります(`biblatex` の場合)。
        \item \citefield{einstein1905annus}{title}
    \end{itemize}
\end{frame}

\section{参考文献リスト}
\begin{frame}{参考文献リスト}
  \printbibliography  
\end{frame}

%% ここまでサンプルのスライド
%%=======================================================================

\end{document}
